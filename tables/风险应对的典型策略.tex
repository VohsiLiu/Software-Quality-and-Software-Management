\vspace{-0.5em}
\begin{spacing}{1.2}
    \centering
    \begin{longtable}{|m{2cm}<{\centering}|m{13cm}|}
        \hline
        \textbf{策略} & \multicolumn{1}{c|}{\textbf{特点}} \\ \hline
        风险转嫁 &
        \vspace{-1.1em}
        \begin{enumerate}[label=,leftmargin=0em]
            \item 指通过某种安排,在放弃部分利益的同时,将部分的项目风险转嫁到其他的团队或者组织
            \item 比如有的公司采取外包的方式,把一部分有技术风险的产品组建交由其他公司开发,在放弃部分收益的同时,也规避了技术风险
            \vspace{-1.1em}
        \end{enumerate}
        \\ \hline
        风险解决 &
        \vspace{-1.1em}
        \begin{enumerate}[label=,leftmargin=0em]
            \item 指采用一些有效措施,使得风险的来源不再存在
            \item 这往往是一种预防性的手段,比如针对项目面临的技术风险,采取技术调研或者引进技术专家的手段,使得原有的风险来源不再存在或者存在可能性极低,从而测试解决该风险
            \vspace{-1.1em}
        \end{enumerate}
        \\ \hline
        风险缓解 &
        指容忍风险的存在,采取一些措施监控风险,不让风险对项目最终目标的实现造成负面影响
        \begin{itemize}
            \item 一般情况下,都需要指定相应的风险缓解计划:理性对待每个关键性的风险,研究可选择的应对方案,并对每个风险皆制定相应的行动过程,是风险缓解计划的关键内容
            \item 特定风险的风险缓解计划包括规避、降低及控制风险发生可能性的技术和方法,或降低风险法身时遭受的损失程度的方法,或上述两者
            \vspace{0.25em}
        \end{itemize}
        监控风险:
        \begin{itemize}
            \item 当风险超过设定的阈值时,实施风险缓解计划,以使受冲击的部分回归到可接受的风险等级
            \item 只有当风险结果评定为高或者无法接受时,才相应指定风险缓解计划和紧急应对计划,其他情况只需要适当监控即可
            \vspace{-1.1em}
        \end{itemize}
        \\ \hline
    \end{longtable}
\end{spacing}
\vspace{-1em}
