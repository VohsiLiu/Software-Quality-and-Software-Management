\vspace{-0.5em}
\begin{spacing}{1.2}
    \centering
    \begin{longtable}{|m{2cm}<{\centering}|m{13cm}|}
        \hline
        \textbf{特征} & \multicolumn{1}{c|}{\textbf{描述}}                                                         \\ \hline
        内聚特征        & 需求规格描述应当尽可能内聚,即仅仅用以说明一件事情                                                                \\ \hline
        完整特征        & 需求规格描述应当完整,不能遗漏信息                                                                        \\ \hline
        一致特征        & 需求规格描述的各个条目和章节不能互相矛盾,需求规格描述与所有外部的参考资料之间也应当消除矛盾之处                                         \\ \hline
        原子特征        & 需求规格描述的过程中,应当尽可能避免连接词的使用。如果需要描述多项内容,可以分别用简单语句加以描述                                        \\ \hline
        可跟踪特征       & 客户需求、产品需求以及产品组件需求必须可以双向跟踪,即客户需求的任何内容,都应当在产品需求和产品组件需求中得到体现。反之,产品组件需求的每一项描述也要可以跟踪到客户需求中的内容 \\ \hline
        非过期特征       & 需求描述的内容必须体现相关干系人对于项目的最新认识。即不能包含已经废弃的需求定义                                                 \\ \hline
        可行性特征       & 需求规格描述的各项内容应该在项目所拥有的资源范围内可以实现                                                            \\ \hline
        强制特征        & 需求规格描述的内容应当体现强制性,即需求规格描述的内容的任何一项缺失,都会导致最终产品不能满足客户期望。因此,可选的需求内容要么不要出现,要么以明确的方式标注          \\ \hline
        可验证特征       & 需求规格描述应当便于在后期开发过程中进行验证。即实现该需求与否,应该有明确的判断标准                                               \\ \hline
    \end{longtable}
\end{spacing}
\vspace{-1em}