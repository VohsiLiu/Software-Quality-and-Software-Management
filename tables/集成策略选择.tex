\vspace{-0.5em}
\begin{spacing}{1.2}
    \centering
    \begin{longtable}{|m{2cm}<{\centering}|m{4.1cm}|m{4.1cm}|m{4.1cm}|}
        \hline
        \textbf{策略名} & \multicolumn{1}{c|}{\textbf{特点}}                               & \multicolumn{1}{c|}{\textbf{优点}}            & \multicolumn{1}{c|}{\textbf{缺点}}                                   \\ \hline
        大爆炸集成策略      & 将所有已经完成的组件放在一起进行一次集成                                           & 需要很少的测试用例                                   & 需要所有有待集成的组件质量非常高,否则会出现难以定位缺陷位置的问题,从而消耗很多测试时间;另外,系统越复杂,规模越大,问题越突出   \\ \hline
        逐一添加集成策略     & 与大爆炸集成策略相反,采取一次添加一个组件的方式进行集成                                   & 很容易定位缺陷位置,特别是在产品组件质量不高的情况下,每次集成之前都有着坚实的质量基础 & 需要测试用例非常多;存在有大量的回归测试,测试时间成本大                                       \\ \hline
        集簇集成策略       & 是对逐一添加集成策略的改进,把有相似功能或者有关联的模块优先进行集成,形成可以工作的组件,然后以组件为单位继续较高层次的集成 & 可以尽早获得一些可以工作的组件,有利于其它组件测试工作的开展              & 过于关注个别组件,而缺乏系统的整体观,不能尽早发现系统层面的缺陷                                   \\ \hline
        扁平化集成策略      & 优先集成高层的部件,然后逐步将各个组件、模块的真正实现加入系统。即尽快构建一个可以工作的扁平化系统              & 可以尽早发现系统层面的缺陷                               & 为了确保完成的系统,需要大量的打“桩”(stub),即提供一些直接提供返回值的伪实现。这种方式往往不能覆盖整个系统应该处理的多种状态 \\ \hline
    \end{longtable}
\end{spacing}
\vspace{-1em}
