\section{软件过程的历史演变和经典工作}

软件发展的两个趋势:
\begin{itemize}
    \item 软件项目规模日益扩大:使得软件越来越难做
    \item 软件在整个系统中的比重日益增加:将软件质量问题的影响上升到前所未有的高度
\end{itemize}

软件危机:指\textbf{落后的软件生产方式}无法满足迅速增长的计算机软件需求,从而导致软件开发与维护过程中出现一系列严重问题的现象。主要体现有:
\vspace{-0.8em}
\begin{multicols}{2}
    \begin{itemize}
        \item 软件开发费用和进度失控
        \item 软件可靠性差
        \item 生产出来的软件难以维护
        \item 用户对“已完成”系统不满意现象经常发现
    \end{itemize}
\end{multicols}
\vspace{-1em}

软件发展的三大阶段:
\begin{itemize}
    \item 软硬件一体化阶段:软件完全依附于硬件,软件作坊(50年代到70年代)
    \item 软件成为独立的产品(70年代到90年代)
    \item 网络化和服务化(90年代中期至今)
\end{itemize}


\subsection{软硬件一体化阶段}

\subsubsection{软件完全依赖于硬件}
软件应用典型特征:软件支持硬件完成计算任务、功能单一、复杂度有限、几乎不需要需求变更。

软件开发典型特征:硬件太贵、团队以硬件工程师和数学家为主。

典型软件过程和实践:非常线性、三思而后行 (measure twice, cut once)

\subsubsection{软件作坊}
软件应用典型特征:功能简单、规模小

软件开发典型特征:很多非专业领域人员涌入软件、高级程序语言出现、质疑权威文化盛行

典型软件过程和实践:Code and Fix、编码 + 改错

\subsubsection{考试题目}
\begin{problem}
	下列软件应用和开发的典型特征中属于软硬件一体化阶段的是?
	\uline{BC}   {\kaishu A. D.属于软件成为独立的产品}
    \vspace{-0.8em}
    \begin{multicols}{2}
        \begin{enumerate}[label=\Alph*.]
            \item 可以通过引入操作系统,摆脱了硬件束缚
            \item 几乎不需要考虑需求变更
            \item 缺乏科班的软件工程师
            \item 系统兼容对应软件开发的成败非常关键
        \end{enumerate}
    \end{multicols}
    \vspace{-1em}
\end{problem}


\subsection{软件成为独立的产品}
软件应用典型特征:摆脱了硬件的束缚(OS)、功能强大、个人电脑出现 $\rightarrow$ 普通人成为软件用户 $\rightarrow$ 需求多变和兼容性要求、来自市场的压力

典型软件过程和实践:
\begin{itemize}
    \item 形式化方法:将所有的方法当作数学方法,做数学化的检验,主要解决质量和正确性问题
    \item 结构化程序设计和瀑布模型:自顶向下,逐步求精
    \item 问题和不足(效率和质量)
    \vspace{-0.8em}
    \begin{multicols}{2}
        \begin{itemize}
            \item 形式化在扩展性和可用性方面存在不足
            \item 瀑布模型成为一个重文档、慢节奏的过程
        \end{itemize}
    \end{multicols}
    \vspace{-1em}
\end{itemize}


\subsection{网络化和服务化}
\subsubsection{网络化和服务化}
软件应用典型特征:功能更复杂、规模更大、用户数量急剧增加、快速演化和需求不确定、分发方法的变化(SaaS)

典型软件过程和实践:
\begin{itemize}
    \item 迭代式(90年代中后期)大型软件系统的开发过程也是一个逐步学习和交流的过程,软件系统的交付不是一次完成,而是通过多个迭代周期,逐步来交付。
    \item 雪鸟会议和敏捷宣言
    \vspace{-0.8em}
    \begin{multicols}{2}
        \begin{itemize}
            \item 个体和互动 胜过 流程和工具
            \item 可以工作的软件 胜过 详尽的文档
            \item 客户合作 胜过 合同谈判
            \item 响应变化 胜过 遵循计划
        \end{itemize}
    \end{multicols}
    \vspace{-1em}
    \vspace{-0.4em}
    \begin{itemize}
        \item 也就是说,尽管右项有其价值,我们更重视左项的价值
    \end{itemize}
    \item XP(eXtreme Programming)方法:偏重于一些工程实践的描述
    \item Scrum: 管理框架和管理实践
    \item KanBan
    \vspace{-0.8em}
    \begin{multicols}{2}
        \begin{itemize}
            \item 精益生产(丰田制造法)的具体实现
            \item 可视化工作流、限定WIP、管理周期时间
        \end{itemize}
    \end{multicols}
    \vspace{-1em}
    \item 开源软件开发方式:
    \begin{itemize}
        \item 一种基于并行开发模式的软件开发的组织与管理方式
        \item Linus 定律:如果有足够多的beta测试者和合作开发者,几乎所有问题都会很快显现,然后自然有人会把它解决
        \item “早发布,常发布,倾听用户的反馈”、“把你的用户当作开发合作者对待,如果想让代码质量快速提升并有效排错,这是最省心的途径”、“设计上的完美不是没有东西可以再加,而是没有东西可以再减”
        \item 代码管理:严格的代码提交社区审核制度、内部开源(inner source)、众包(Crowdsourcing)
    \end{itemize}
\end{itemize}

\subsubsection{更深化的网络化和服务化}
软件应用典型特征:
\begin{itemize}
    \item 进一步服务化和网络化(移动是主流)随处可见(pervasive)
    \item 用户需求的多样性进一步凸显
    \item 软件产品和服务的地位变化
    \item 错综复杂的部署环境
    \item 近乎苛刻的用户期望
    \vspace{-0.8em}
    \begin{multicols}{2}
        \begin{itemize}
            \item 多:功能丰富,个性化
            \item 快:快速使用,及时更新,快速解决问题
            \item 好:稳定,可靠,安全,可信
            \item 省:用户的获得成本低,最好免费
        \end{itemize}
    \end{multicols}
    \vspace{-1em}
\end{itemize}

软件开发的典型特征:
\begin{itemize}
    \item 空前强大的开发和部署环境:XaaS,IaaS、PaaS、SaaS
    \item 盛行开源和共享文化
    \item 盛行敏捷开发
    \item 软件工程的潜在支撑力量获得了长足进步(AI、大数据、云计算)
\end{itemize}

典型软件过程和实践:DevOps
\vspace{-0.8em}
\begin{multicols}{2}
    \begin{itemize}
        \item 开发运维一体化
        \item 方法论的基础是软件敏捷开发、精益思想和看板 KanBan 方法
        \item 以领域驱动设计为指导的微服务架构方式
        \item 大量虚拟化技术的使用
        \item 一切皆服务 XaaS 的理念指导
        \item 构建了强大的工具链,支持高水平自动化
    \end{itemize}
\end{multicols}
\vspace{-1em}

\subsection{考试题目}
\begin{problem}
软件发展三大阶段的特点和主流开发方法
\begin{itemize}
    \item 软硬件一体化
    \begin{itemize}
        \item 特点:软件支持硬件完成计算任务、功能单一、复杂度有限、几乎不需要需求变更
        \item 主流开发方法:三思而后行;Code and Fix;编码 + 改错
    \end{itemize}
    \item 软件成为独立的产品
    \begin{itemize}
        \item 特点:摆脱了硬件的束缚(操作系统)、功能强大、个人电脑出现、需求多变、兼容性要求、来自市场的压力
        \item 主流开发方法:形式化方法、结构化程序设计和瀑布模型
    \end{itemize}
    \item 网络化和服务化
    \begin{itemize}
        \item 特点:功能更复杂、规模更大、用户数量急剧增加、快速演化和需求不确定、分发方式的变化、进一步的服务化和网络化、盛行开源和共享文化
        \item 主流开发方法:迭代式开发、敏捷开发(XP、Scrum、Kanban)、开源软件开发方式、DevOps
    \end{itemize}
\end{itemize}
\end{problem}

\begin{problem}
请结合软件发展的三大阶段,描述不同阶段的典型软件开发方法和实践
\begin{itemize}
    \item 软硬件一体化:线性顺序过程、事实上是硬件开发流程,measure twice, cut once
    \item 软件成为独立产品:结构化程序设计、瀑布生命周期模型、成熟度运动
    \item 网络化和服务化:迭代式开发、敏捷运动
\end{itemize}
\end{problem}




