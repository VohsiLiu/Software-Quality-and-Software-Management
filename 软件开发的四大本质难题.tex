\section{软件开发的四大本质难题}

\subsection{四大本质难题是什么}
\textbf{不可见性}
\begin{itemize}
    \item 软件是一种“看不见、摸不着”的逻辑实体、不具有空间的形体特征
    \item 开发人员可以直接看到程序源代码,但是源代码本身并不是软件本身
    \item 软件是以机器代码的形式运行,但是开发人员无法看到源代码是如何运行的
\end{itemize}

\textbf{复杂性}
\begin{itemize}
    \item 对于软件复杂的需求导致了软件的复杂性
\end{itemize}

\textbf{可变性}
\begin{itemize}
    \item 软件的变化(随时间推移)对其失效率的影响,软件的可变性体现在软件本身升级,功能的变化等
\end{itemize}

\textbf{一致性}
\begin{itemize}
    \item 软件不能独立存在,要依附于一定的环境(如硬件、网络、以及其他软件)
    \item 软件必须遵循从人为的惯例并适应已有的技术和系统
    \item 软件需要随从接口不同而变化,随着时间推移而变化,而这些变化是不同人设计的结果
\end{itemize}


\subsection{四大本质难题之间的关系}
\vspace{-0.8em}
\begin{multicols}{2}
    \begin{itemize}
        \item 除不可见性外,其他三个本质难题因项目而异
        \item 四大本质难题互相促进
        \item 本质难题变化带动软件方法(过程)演变
    \end{itemize}
\end{multicols}
\vspace{-1em}


\subsection{几个注意点}
\begin{itemize}
    \item 软件开发四大本质难题永远存在,不可能彻底解决,在不同时期凸显程度有差异
    \item 软件开发本质上是智力劳动,开发者心理方面的因素不可忽视
\end{itemize}


\subsection{考试题目}
\begin{problem}
	关于 Brooks 提及的软件开发本质难题,下列说法中不准确的是:
	\uline{AB}    
    \begin{enumerate}[label=\Alph*.]
        \item 本质难题总共有四个,分别为复杂、不可见、可变和质量挑战
        \item 既然是本质难题,那就说明是根植于软件开发本身,因而是不可能在软件开发当中得到
            缓解
        \item 严格来说,只有不可见才是真正的“本质”难题,其他三个因项目而异
        \item 四大本质难题贯穿软件发展的不同历史阶段,但是在不同历史阶段,相互凸显程度不一
            样
    \end{enumerate}
\end{problem}
